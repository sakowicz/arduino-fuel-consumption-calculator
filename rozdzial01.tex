\chapter{Wstęp}
\section{Motywacja}
Starsze samochody nie posiadają wskaźników czy komputerów pokładowych, które informują na temat poziomu spalania pojazdu. Wiele z nich nie posiada również złącz diagnostycznych, które umożliwiają wyprowadzenie takich danych na zewnątrz. Posiadając podstawową wiedzę na temat działania silnika spalinowego oraz posługując się obliczeniami matematycznymi można wyprowadzić te informacje dla użytkownika samochodu. Z pomocą przychodzą mikrokontrolery.
\section{Cel}
Celem pracy jest stworzenie oraz implementacja urządzenia do wyświetlania aktualnego oraz średniego spalania pojazdu. Projekt ma wykorzystywać mikrokontroler Arduino oparty na architekturze AVR.
\\Celem pobocznym jest wykonanie przyjaznego interfejsu umożliwiającego przekazanie tych informacji użytkownikowi pojazdu. Dodatkowo system powinien wyświetlać informacje na temat prędkości, aktualnej temperatury, aktualnego poziomu napięcia w układzie pojazdu, aktualnego położenia pedału gazu oraz zapisywać archiwalne dane dotyczące spalania w pamięci nieulotnej.
\section{Zakres pracy}

\section{Układ pracy}

