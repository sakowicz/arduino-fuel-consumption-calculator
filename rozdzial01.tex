\chapter{Wstęp}
\par     Starsze samochody nie posiadają wskaźników czy komputerów pokładowych, które przekazują informacje na temat poziomu spalania paliwa. Wiele z nich nie posiada również złącz diagnostycznych, które umożliwiają wyprowadzenie takich danych na zewnątrz. Posiadając podstawową wiedzę na temat działania silnika spalinowego oraz posługując się obliczeniami matematycznymi można wyprowadzić te informacje dla użytkownika samochodu. Do tego celu można zastosować mikrokontroler.
\par Celem pracy jest opracowanie oraz implementacja modułu do wyświetlania aktualnego oraz średniego spalania pojazdu. Projekt ma wykorzystywać mikrokontroler Arduino oparty na architekturze AVR.
\par Celem pobocznym jest wykonanie przyjaznego interfejsu umożliwiającego przekazanie tych informacji użytkownikowi pojazdu. Dodatkowo system powinien wyświetlać informacje na temat prędkości, aktualnej temperatury, aktualnego poziomu napięcia w układzie pojazdu, aktualnego położenia pedału gazu oraz zapisywać archiwalne dane dotyczące spalania w pamięci nieulotnej.
\par W kolejnym rozdziale opisana jest ogólna zasada działania oraz założenia projektu, określono teoretyczne i matematyczne aspekty pracy. W trzecim rozdziale opisana jest fizyczna implementacja projektu, tj. opis poszczególnych modułów oraz wyprowadzeń. Rozdział czwarty obejmuje programowanie modułu, tj. zastosowanie matematycznych obliczeń w kodzie programu oraz konfigurowanie i inicjalizacja modułu. W rozdziale piątym opisano programowanie interfejsu oraz działanie programu, na które ma wpływ użytkownik. Ostatnim rozdziałem jest podsumowanie, w którym oprócz opisania wykonanych prac zawarto możliwości dalszej rozbudowy.