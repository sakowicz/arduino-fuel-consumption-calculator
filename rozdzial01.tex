\chapter{Wstęp}
\section{Motywacja}
Starsze samochody nie posiadają wskaźników czy komputerów pokładowych, które informują na temat poziomu spalania pojazdu. Wiele z nich nie posiada również złącz diagnostycznych, które umożliwiają wyprowadzenie takich danych na zewnątrz. Posiadając podstawową wiedzę na temat działania silnika spalinowego oraz posługując się obliczeniami matematycznymi można wyprowadzić te informacje dla użytkownika samochodu. Z pomocą przychodzą mikrokontrolery.
\section{Cel}
Celem pracy jest stworzenie oraz implementacja urządzenia do wyświetlania aktualnego oraz średniego spalania pojazdu. Projekt ma wykorzystywać mikrokontroler Arduino oparty na architekturze AVR.
\\Celem pobocznym jest wykonanie przyjaznego interfejsu umożliwiającego przekazanie tych informacji użytkownikowi pojazdu. Dodatkowo system powinien wyświetlać informacje na temat prędkości, aktualnej temperatury, aktualnego poziomu napięcia w układzie pojazdu, aktualnego położenia pedału gazu oraz zapisywać archiwalne dane dotyczące spalania w pamięci nieulotnej.
\section{Zakres i układ pracy}
W rozdziale drugim opisana jest ogólna zasadza działania oraz założenia projektu. W tym dziale określono teoretycznie i matematycznie w jaki sposób obliczane są potrzebne dane. W trzecim rozdziale opisana jest fizyczna implementacja projektu: w jaki sposób podłączone są przewody, w jaki sposób działają moduły oraz w jakie układy są połączone. Rozdział czwarty obejmuje programowanie modułu, tj. zastosowanie matematycznych obliczeń w kodzie programu oraz konfigurowanie i inicjalizacja modułu. W rozdziale piątym opisano programowanie interfejsu oraz działanie programu, na które ma wpływ użytkownik. Ostatnim rozdziałem jest podsumowanie, w którym opisano również możliwości rozbudowy.