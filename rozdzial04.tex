\chapter{Programowanie i konfiguracja modułu}
\section{Wprowadzenie}
\subsection{Środowisko programistyczne}

Domyślnym środowiskiem programistycznym (IDE) jakie udostępnia Arduino jest Arduino IDE i to właśnie w nim tworzony oraz kompilowany będzie kod źródłowy mikrokontrolera.

\subsection{Przyjęte nazewnictwo podczas pisania kodu}
W obrębie tego projektu zostało postanowione, że nazwy zmiennych będą zapisywane w języku polskim. Zwykłe zmienne będą nazywane w konwencji \texttt{snake\_case} a nazwy funkcji w \texttt{camelCase}. Dodatkowo wprowadzono znacznik ---, który oznacza, że w tym miejscu znajduje się jakiś kod, jednak jest on pominięty w listingu, ponieważ nie wnosi nic do aktualnego przykładu.

\subsection{Koncepcja programowania modułu Arduino}
Strukturę kodu źródłowego można podzielić na dwie główne części: blok ustawień (setup) oraz główną pętle programu (loop).
\begin{itemize}
\item{sekcja, a w zasadzie funkcja \texttt{setup()} wykonuje się tylko raz i służy do ustawiania trybów pinów, inicjalizacji zmiennych, importowania bibliotek, ustawiania przerwań itp.}
\item{główna funkcja programu, a konkretniej \texttt{loop()} jest pętlą, która z założenia wykonuje się w nieskończoność, więc to w niej będą wykonywać się obliczenia, wyświetlanie danych itp.}
\end{itemize}

\subsection{Przerwania}
\subsubsection{Sprzętowe}
Jednym z najważniejszych zagadnień, bez których zbudowanie tego projektu byłoby znacznie utrudnione, są przerwania. Przerwania są to sygnały, które jak sama nazwa wskazuje powodują przerwanie wykonywania głównej pętli programu i wykonują zadane polecenia. W tym projekcie będą używane dwa przerwania: pierwsze służące do reagowania na sygnał z VSS oraz drugie, które będzie przesyłać informacje o włączeniu/wyłączeniu wtrysku.\\

Piny przerwań są to cyfrowe piny wejściowe, które mogą mieć tylko dwa stany \texttt{LOW} lub \texttt{HIGH}\\

Arduino Leonardo posiada 4 ustawienia przerwań \cite{ard_ref}
\begin{itemize}
    \item \texttt{LOW} - aktywowane jest wtedy, kiedy stan pinu jest LOW,
    \item \texttt{CHANGE} - aktywowane jest wtedy, kiedy stan zmienia się,
    \item \texttt{RISING} - aktywowane jest wtedy, kiedy stan zmienia się z \texttt{LOW} na \texttt{HIGH} 
    \item \texttt{FALLING} - aktywowane jest wtedy, kiedy stan zmienia się z \texttt{HIGH} na \texttt{LOW} 
\end{itemize}

\subsubsection{Oparte na czasie}
Oprócz przerwań opartych na stanach cyfrowych pinów potrzeba również przerwanie, które wykonuje się co określony czas, aby móc przetworzyć i wyświetlić informację. Arduino w swojej dokumentacji nie posiada informacji na temat takich przerwań, jednak jako, iż oparte jest ono na architekturze AVR, która takie przerwania posiada, zostaną one wykorzystane.\\
Zostało założone, że częstotliwość obliczania oraz wyświetlania informacji to 1 sekunda. W tym celu zostanie stworzone przerwanie, które będzie się opierać na dość dokładnym pomiarze czasu, który bazuje na ilości taktów procesora.\\

\section{Implementacja}
\subsection{Przerwanie timera} \label{setting_timer_interupt}

W poniższym listingu został pokazany kod, który korzysta z Timera 1, który wywołuje przerwanie co 1/4 sekundy. Instrukcje, które zostaną wprowadzone w funkcji \texttt{coSekunde()} będą się wykonywać co dokładnie jedną sekundę.

\begin{lstlisting}[label=list:timer_int,caption=Ustawianie przerwania timera,
basicstyle=\footnotesize\ttfamily]
volatile int licznik_przerwania;
---
void coSekunde()
{
    // Instrukjce w tym miejscu będą wykonywane co sekudnę
}

// Przerwanie wykonywane co 1/4 sekundy
ISR(TIMER1_OVF_vect)
{
    licznik_przerwania++;
    TCNT1 = 3036;
    
    if(licznik_przerwania > 3) 
    {
        coSekunde();

        TCNT1 = 3036;
        licznik_przerwania = 0;
    }
}
---
void setup()
{
---
    TCCR1A = 0x00;
    TCNT1 = 3036;
    TCCR1B |= ((1 << CS10) | (1 << CS11));
    TIMSK1 |= (1 << TOIE1);
---
}
---
\end{lstlisting}



\subsection{Aktualna prędkość}
\subsubsection{Konfiguracja przerwania}

Aby obliczyć prędkość zostanie wykorzystane przerwanie typu FALLING, aby każdy z sygnałów został zliczony.

Jak było przestawione w \ref{vss} sygnał od VSS jest podłączony do pinu 7, który obsługuje przerwanie 4.

\begin{lstlisting}[label=list:vss_int,caption=Ustawianie przerwania VSS,
basicstyle=\footnotesize\ttfamily]
volatile int licznik_impulsow;

const vss_pin_int = 4;
---
void liczenieImpulsow()
{
    licznik_impulsow++;
}
---
void setup()
{
---
    attachInterrupt(vss_pin_int, liczenieImpulsow, FALLING);
---
}
---
\end{lstlisting}
\subsubsection{Obliczenia} \label{code_speed}

Aby wyliczyć aktualną prędkość zostaną zastosowane obliczenia z \ref{calc_speed} oraz funkcja \texttt{coSekunde()} z \ref{setting_timer_interupt}.

\begin{lstlisting}[label=list:vss_int,caption=Wyliczanie aktualnej prędkości,
basicstyle=\footnotesize\ttfamily]
volatile int impulsy_vss;
float predkosc;
const float mila = 1.609344;
---
void coSekunde()
{
    predkosc = (mila/4000.0)*impulsy_vss*3600;
    impulsy_vss = 0;
}
---
\end{lstlisting}

\subsection{Aktualne spalanie}
\subsubsection{Konfiguracja przerwania}

Aby obliczyć aktualne spalanie zostanie wykorzysane przerwanie typu \texttt{CHANGE}. W ten gdy pin zmieni swój stan zostanie mierzony czas otwarcia wtrysku w mikrosekundach

Jak było przestawione w \ref{inj} sygnał od wtrysku jest podłączony do pinu 2, który obsługuje przerwanie 1.

\begin{lstlisting}[label=list:inj_int,caption=Ustawianie przerwania wtrysku,
basicstyle=\footnotesize\ttfamily]

volatile unsigned long czas_stanu_low;
volatile unsigned long czas_stanu_high;
volatile unsigned long czas_otwarcia_wtrysku;

const inj_pin = 2;
const inj_pin_int = 1;
---
void liczenieCzasuOtwarciaWtrysku()
{
    if (digitalRead(inj_pin) == LOW)
    {
        czas_stanu_low = micros();
    }
    if (digitalRead(inj_pin) == HIGH)
    {
        czas_statnu_high = micros();
        czas_otwarcia_wtrysku += czas_stanu_high-czas_stanu_low;
    }
}
---
void setup()
{
---
    attachInterrupt(inj_pin_int, liczenieCzasuOtwarciaWtrysku, CHANGE);
---
}
---
\end{lstlisting}

\subsubsection{Obliczenia}

Aby wyliczyć aktualne spalanie zostaną zastosowane obliczenia z \ref{calc_consumption}, funkcja \texttt{coSekunde()} z \ref{setting_timer_interupt} oraz wyliczona już prędkość z poprzedniego podrozdziłu \ref{code_speed}.

\begin{lstlisting}[label=list:vss_int,caption=Obliczanie aktualnego spalania,
basicstyle=\footnotesize\ttfamily]
float predkosc;
float spalanie_s;
float spalanie_h;
float spalanie_100;

const stala_wtrysku = 190;

---
void coGodzine()
{
---
    spalanie = ((stala_wtrysku/60*(czas_otwarcia_wtrysku/1000000.0))*4.0);
    spalanie_h = ((spalanie/1000.0)*3600.0);
    spalanie_100 = ((100*spalanie_h)/predkosc);
---
}


\end{lstlisting}


\subsection{Położenie pedału gazu}

Jako, że sygnał położenia pedału gazu działa niczym potencjometr, do obliczenia jego procentowej wartości została użyta funkcja \texttt{analogRead}, która ma rozdzielczość 10 bit i zwraca liczbę całkowitą z przedziału 0-1023 w zależności od odczytu na wejściu \cite{ard_ref}.

\begin{lstlisting}[label=list:code_thr,caption=Obliczanie procentowego nacisku na pedał gazu,
basicstyle=\footnotesize\ttfamily]

const thr_pin = 0;

---
float aktualnyNacisk()
{
    int odczyt;
    
    // Przeskalowany, ponieważ sygnał nigdy nie równa się 5V
    odczyt = analogRead(thr_pin)-88.0; 
    
     // Zwraca nacisk w procentach
    return odczyt/(1023-88) * 100;
}
---
\end{lstlisting}


\subsection{Odczyt woltomierza}

Jak zostało wspomniane w założeniach, aplikacja ma również informować użytkownika o aktualnym napięciu w układzie samochodu. W tym celu zostały użyte dwa rezystory (\ref{voltometer}).

\begin{lstlisting}[label=list:code_thr,caption=Obliczanie aktualnego napięcia,
basicstyle=\footnotesize\ttfamily]

const voltometer_pin = 1;

float R1 = 100000.0; // Opór rezystora R1
float R2 = 10000.0; // yOpór rezystora R2

---
float woltomierz()
{
    int odczyt = analogRead(voltometer_pin);
    float napiecie_na_wyjsciu = (odczyt * 5.0) / 1024.0;
    float wartosc_napiecia = napiecie_na_wyjsciu / (R2/(R1+R2));
    
    // Zabezpieczenie, aby nie wyświetlać pomiaru w granicy błędu
    if (wartosc_napiecia<0.09)
    {
    	return 0.0;
    }
    
    return wartosc_napiecia;
}
---
\end{lstlisting}


\subsection{Mikro przełącznik}

\subsubsection{Ustawianie trybu podciągającego}

Aby przełącznik działał poprawnie należałoby użyć rezystora podciągającego, żeby uniknąć przypadków, że na wejściu będzie panował stan nieznany. Arduino Leonardo umożliwia ustawienie pinu w tryb \texttt{PULLUP}, dzięki czemu nie potrzeba dokładać kolejnego rezystora. 
\begin{lstlisting}[label=list:code_pullup,caption=Ustawianie pinu switcha w tryb PULLUP,
basicstyle=\footnotesize\ttfamily]

const switch_pin = 13;

void setup()
{
    pinMode(switch_pin,INPUT_PULLUP);
    digitalWrite(switch_pin, HIGH);
}
---
\end{lstlisting}

\subsubsection{Drgania mikro przełącznika}
Większość mikro przełączników mechanicznych posiada problem z drganiami styków. W przypadku, gdy przycisk zostaje wciśnięty lub puszczony jego styki potrafią drgać powodując zmiane stanu wejściowego na pinie. Aby się przed tym zabezpieczyć została napisana funkcja, która po zmianie stanu na wejściu czeka określoną ilość czasu i sprawdza raz jeszcze stan przycisku w celu uniknięcia błędnego wykonywania kodu. Dla tego typu mikro przełączików przyjęło się, iż ten okres czasu wynosi 20 \mu s.\\

Aby polepszyć interakcję użytkowników z mikro kontrolerem zostały wprowadzone dwa tryby klikania w switch: zwykłe kliknięcie oraz przytrzymanie (min. 1 sekundę).

\begin{lstlisting}[label=list:code_switch,caption=Obsługa mikroprzełacznika,
basicstyle=\footnotesize\ttfamily]

#define czas_drgan 20
#define czas_przytrzymania 1000

const switch_pin = 13;

int switch_wartosc = 0;
int switch_ostatnia_wartosc = 0;
long czas_przycisku_wcisnietego;
long czas_przycisku_puszczonego;
boolean ignoruj_nacisniecie = false; 


void nacisnieciePrzycisku()
{
    // To wykona się po naciśnięciu przycisku
}
void przytrzymaniePrzycisku()
{
    // To wykona się po przytrzymaniu przycisku
}
void mirkroPrzelacznik()
{

	switch_wartosc = digitalRead(switch_pin);

    if (
        switch_wartosc == LOW && 
        switch_ostatnia_wartosc == HIGH && 
        (millis() - czas_przycisku_puszczonego) > long(czas_drgan)
    )
    {
    	czas_przycisku_wcisnietego = millis();
    }
    
    if (
        switch_wartosc == HIGH &&
        switch_ostatnia_wartosc == LOW &&
        (millis() - czas_przycisku_wcisnietego) > long(czas_drgan))
    {
    	if (ignoruj_nacisniecie == false)
    	{
    		nacisnieciePrzycisku();
    	}
    	else
    	{
    	    ignoruj_nacisniecie = false;
    	}
    	czas_przycisku_puszczonego = millis();
    }
    
    if (
        switch_wartosc == LOW &&
        (millis() - czas_przycisku_wcisnietego) > long(czas_przytrzymania))
    {
    	przytrzymaniePrzycisku();
    
    	ignoruj_nacisniecie = true;
    	czas_przycisku_wcisnietego = millis();
    }
    
    switch_ostatnia_wartosc = switch_wartosc;
}
---
\end{lstlisting}

\subsection{Wyświetlacz LCD}

Jako iż wyświetlacz jest zgodny ze sterownikiem HD44780 została użyta biblioteka \texttt{LiquidCrystal.h} \cite{lib_lcd}, która umożliwia łatwą komunikację z tym LCD.

\begin{lstlisting}[label=list:lcd_setup,caption=Inicjalizacja wyświetlacza,
basicstyle=\footnotesize\ttfamily]
#include <LiquidCrystal.h>
---
void setup()
{
---
    LiquidCrystal lcd(12,11,6,5,4,3);
---
}
---
\end{lstlisting}

\subsection{Czujnik temperatury}

Aby skorzystać z cyfrowego czujnika temperatury działającego na interfejsie 1-Wire zostały dołączo e dwie biblioteki: \texttt{DallasTemperature.h} \cite{lib_dallas}, która odpowiada za odczyt i konfiguracje czujnika oraz \texttt{OneWire.h} \cite{lib_onewire}, dzięki której można korzytać z interfejsu 1-Wire.

\begin{lstlisting}[label=list:d18s20_setup,caption=Inicjalizacja czujnika i odczyt temperatury,
basicstyle=\footnotesize\ttfamily]
#include <OneWire.h>
#include <DallasTemperature.h>

#define ONE_WIRE_BUS 10

OneWire oneWire(ONE_WIRE_BUS);
DallasTemperature sensors(&oneWire);

---
void setup()
{
---
    sensors.begin();
---
}

float temperatura() 
{
    sensors.requestTemperatures();
    return sensors.getTempCByIndex(0);
}
---
\end{lstlisting}


\subsection{Pamięć nieulotna EEPROM}
Pamięć EEPROM (ang. Electrucally-Erasable Programmable Read-Only Memory) jest to rodzaj pamięci nieulotnej. Można z niej odczytywać nieograniczoną ilość razy, jednak ilość zapisań/kasowań jest ograniczona. W przypadku Arduino w wersji Leonardo ograniczenie to wynosi 100000 cykli. Wielkość pamięci w tej wersji płytki wynosi 1 kB, przy czym jedna komórka pamięci może mieć wartość od 0 do 255 bitów.\\
Aby móc zapisywać/odczytywać dane z EEPROM została użyta biblioteka \texttt{EEPROM.h} \cite{lib_eeprom}. Dostarcza ona m.in metody, które umożliwiają zapis tj. \texttt{EEPROM.write()} oraz odczyt tj. \texttt{EEPROM.read()}.
\subsubsection{Operacje na większych danych}
W przypadku tego projektu dane, które są zapisywane to przejechany dystans oraz spalone paliwo. Aby móc zapisać i odczytać tak duże wartości zostały zaimplementowane funkcje \texttt{EEPROMZapiszLong()} oraz \texttt{EEPROMOdczytajLong()}, które umożliwiają zapis 32 bitowego typu \texttt{long} na czterech komórkach pamięci umożliwiających zapis 8 bitowych danych.

\begin{lstlisting}[label=list:eeprom_long,caption=Zapis i odczyt dużych liczb do/z EEPROM,
basicstyle=\footnotesize\ttfamily]
#include <EEPROM.h>
---
void EEPROMZapiszLong(int adres, long wartosc)
{
    byte czwarty = (wartosc & 0xFF);
    byte trzeci = ((wartosc >> 8) & 0xFF);
    byte drugi = ((wartosc >> 16) & 0xFF);
    byte pierwszy = ((wartosc >> 24) & 0xFF);
    
    EEPROM.write(adres, czwarty);
    EEPROM.write(adres + 1, trzeci);
    EEPROM.write(adres + 2, drugi);
    EEPROM.write(adres + 3, pierwszy);
}

long EEPROMOdczytajLong(long adres)
{
    long czwarty = EEPROM.read(adres);
    long trzeci = EEPROM.read(adres + 1);
    long drugi = EEPROM.read(adres + 2);
    long pierwszy = EEPROM.read(adres + 3);
    
    return 
    ((czwarty << 0) & 0xFF)
    + ((trzeci << 8) & 0xFFFF)
    + ((drugi << 16) & 0xFFFFFF) 
    + ((pierwszy << 24) & 0xFFFFFFFF);
}
\end{lstlisting}

Zostało założone, iż ilość spalonego paliwa jest zapisywana od zerowej komórki pamięci, a przejechany dystans od czwartej. To oznacza, że aby odczytać zapisane dane należy wywołać funkcje \texttt{EEPROMOdczytajLong(0)}, aby odczytać spalone paliwo oraz \texttt{EEPROMOdczytajLong(4)}, aby odczytać przejechany dystans. Analogicznie jest z zapisem.

\subsubsection{Ograniczenia i optymalizacja}

Jak zostało wspomniane wcześniej ilość zapisów do tej pamięci jest ograniczona. Aby wydłużyć jej żywotność została wprowadzona optymalizacja zapisu do pamięci. Zapis aktualnych danych jest wykonywany tylko, gdy samochód stoi (czyli prędkość pojazdu jest mniejsza niż 2 km/h) oraz gdy dystans od poprzedniego zapisu jest większy niż kilometr. Takie podejście znacznie zredukowało liczbę zapisów.