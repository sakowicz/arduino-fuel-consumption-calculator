\chapter{Założenia i zasada działania}
\section{Założenia}
\subsection{Pojazd}
Projekt będzie tworzony przykładzie pojazdu Honda Civic VI z silnikiem D14A8 z roku 1998.

\section{Zasada działania}
\subsection{Potrzebne dane}
Do odczytania informacji potrzebnych do obliczenia aktualnego spalania potrzebujemy dwóch głównych informacji:
\begin{itemize}
\item{aktualnej prędkości pojazdu}
\item{ilości wstrzykiwanego paliwa na jednostkę czasu}
\end{itemize}

\subsection{Obliczanie aktualnej prędkości pojazdu}
Do tego celu potrzebny będzie sygnał, który wysyła VSS (Vehicle Speed Sensor) - jest to urządzenie, z którego korzysta wskaźnik prędkości na desce rozdzielczej. W ogólnym założeniu działa on tak, iż wysyła określoną ilość impulsów co określony dystans. W przypadku pojazdu, na którym budowany jest projekt jest to \textbf{4000 impulsów} co \textbf{jedną milę}.\\
\\Jeżeli prędkość będziemy odświeżać co sekundę, oraz po odświeżeniu będziemy resetować ilość impulsów VSS to prędkość obliczymy z wzoru (zobacz równanie (\ref{eq:speed})).

\begin{equation}\label{eq:speed}
V[km/h] = \frac{S[km]}{I} 3600 i
\end{equation}
gdzie:
\begin{description}
\item[S] długość jednostki dystansu, w przeliczeniu na kilometry, w której VSS wysyła sygnał
\item[I] to stała liczba impulsów VSS, co którą pojazd pokonuje określony dystans
\item[i] to liczba impulsów w tej sekundzie
\end{description}
W przypadku tego pojazdu wzór przedstawia się następująco (zobacz równanie (\ref{eq:speed2})).
\begin{equation}\label{eq:speed2}
V[km/h] = \frac{1.609344[km]}{4000} 3600i
\end{equation}

\subsection{Obliczanie spalania}
Koncepcja obliczania spalania polega na mierzeniu czasu otwarcia wtrysku, czyli czasu, w którym paliwo jest wstrzykiwane do komory spalania silnika. Jeśli znamy stałą wtryskiwacza, czyli parametr mówiący jaka ilość paliwa jest wtryskiwana w danym okresie czasu, możemy obliczyć aktualne spalanie samochodu.\\
Pojazd, na którym oparta jest ta praca posiada wtryski 190cc, co oznacza, że wtryskują one \textbf{190 ml paliwa na minutę}.

\subsubsection{Obliczanie spalania na godzinę}
W tym wypadku również, jak w przypadku prędkości, co jedną sekundę sprawdzamy, przez jaki czas wtrysk wstrzykiwał paliwo. Mając już tę informację spalanie na godzinę wyliczamy ze wzoru(\ref{eq:consumptionPerHour}))

\begin{equation}\label{eq:consumptionPerHour}
C_{h}[L/h] = N\frac{I[ml]}{60}\frac{i[\mu s]}{1000000}\frac{3600}{1000}
\end{equation}
gdzie:
\begin{description}
\item[N] to ilość wtrysków w silniku
\item[I] to stała wtrysku
\item[i] to czas otwarcia wtrysku w danej sekundzie
\end{description}
Należałoby tutaj zaznaczyć, iż w tym projekcie pobierana jest informacja tylko z jednego wtrysku. Każdy wtrysk wstrzykuje za każdym razem tyle samo paliwa, więc ewentualny błąd przybliżenia do N wtrysków jest pomijalny.
\subsubsection{Obliczanie aktualnego spalania na 100 km}
Bardziej przydatnym, z perspektywy odbiorcy, formatem aktualnego spalania jest spalanie zależne od aktualnej prędkości, czyli spalanie na 100 kilometrów. Zostaje ono obliczone ze wzoru (\ref{eq:consumptionPer100}))

\begin{equation}\label{eq:consumptionPer100}
C_{100km}[L] = 100\frac{C_{h}[L/h]}{V[km/h]}
\end{equation}
gdzie:
\begin{description}
\item[C_{h}] to spalanie na godzinę
\item[V] to prędkość
\end{description}\\